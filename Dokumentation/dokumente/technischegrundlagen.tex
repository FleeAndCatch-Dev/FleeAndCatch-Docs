\section{Technische Grundlagen}
\subsection{Robotik}

\subsubsection{Grundlagen}
\subsubsection{Mobile Roboter}
\subsubsection{Antriebsarten}
\subsubsection{Sensorik}
\subsubsection{LEGO Mindstorm}

\newpage
\subsection{App Entwicklung}

Dieses Kapitel beschreibt die Entwicklung einer App und geht dabei auf die Grundlagen, sowie speziell auf plattformübergreifende Möglichkeiten ein.

\subsubsection{Grundlagen}

Eine App, steht kurz für Application. Sie steht für eine Anwendung auf einem mobilen Gerät, wie einem Smartphone, oder Tablet. Eine App

\subsubsection{Arten}

\subsubsection*{Web Apps}
\subsubsection*{Native Apps}
\subsubsection*{Hybride Apps}

\subsubsection{Plattformübergreifende Programmierung}

\subsubsection*{Qt}

Qt ist ein plattformübergreifendes Framework zur Entwicklung von Anwendungen auf PC, Embedded und mobilen Geräten. Die Implementierung findet dabei mit der Programmiersprache C++ statt, wodurch Betriebssysteme, wie Linux, OS X, Windows und BlackBerry unterstützt werden.\\
Das Prinzip der plattformübergreifenden Entwicklung in Qt basiert auf Standard Compilern, wie CLang, oder GNU, die dafür sorgen, das der implementierte Qellcode auf verschiedenen Systemen übersetzt werden kann. Qt liefert unter anderem eine Klassenbibliothek, die es ermöglicht bereits implementierten Quellcode zu nutzen und dem Entwickler somit Zeit abnehmen.\\

\subsubsection*{Apache Cordova}

Apache Cordova ist ein Open-source Entwicklungs-Framework, das auf der technischen Möglichkeit von Hybrid Anwendungen basiert. Es erlaubt die Implementierung der Anwendungen mitstandardisierten Web-Technologien, wie HTML5, CSS3 und JavaScript.\\


\subsubsection*{Xamarin}
\subsubsection{Mono}
\subsubsection{.Net Framework}

\subsection{Java}

\subsubsection{Grundlagen}
\subsubsection{Java Runtime Environment}

\subsection{Kommunikation}

\subsubsection{Grundlagen}
\subsubsection{Wifi}
\subsubsection{Datenaustauschformate}
\subsubsection{JSON}

\subsection{Komponenten}

\subsubsection{EV3}
\subsubsection{Raspberry Pi}