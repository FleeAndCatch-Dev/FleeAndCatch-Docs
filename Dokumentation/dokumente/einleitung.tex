\section{Einleitung}

Heutzutage werden viele Arbeitsschritte in der Produktion, als auch Dienstleistungen von Maschinen verrichtet, da diese effizienter Arbeiten und weniger Kosten als Menschen verursachen. Da jede Maschine auf einen spezifischen Arbeitsschritt konfiguriert ist, müssen die verschiedenen Maschinen untereinander wie ein Schwarm agieren. Diese Verhaltensstrukturen kommen ursprünglich aus dem Tierreich, wie Fischschwärme, Ameisen oder Bienen. Hierbei erledigt jedes Individuum seine zugewiesenen Aufgaben und hält die anderen Parteien auf dem aktuellen Stand.\\
In diesem Projekt werden diese Verhaltensmuster aus dem Tierreich aufgegriffen und anhand eines Verhaltensszenarios mit Kleinrobotern verwirklicht, die autonom agieren und kommunizieren, um zusammen ihr Ziel zu erreichen.\\