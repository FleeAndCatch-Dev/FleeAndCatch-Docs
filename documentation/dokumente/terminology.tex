%###########################################################################################################################################
%### TERMINILOGIE FESTLEGUNG ###############################################################################################################
%###########################################################################################################################################

%### Beschreibung ###################################################################################%
%# Dieses Dokument dient zur Festlegung der Terminoligie. Hier können Begriffe die bestimmte Kompo- #%
%# nenten, Systeme oder Sachverhalte beschreiben festgelegt und Variablen zugeordnet werden. Die    #%
%# Verwendung dieser Variablen in der Ausarbeitung soll sicherstellen das für eine Komponente, ein  #%
%# System oder Sachverhalt immer der selbe Begriff verwendet wird.                                  #%
%#                                                                                                  #%
%# Beispiel:                                                                                        #%
%# Für das (Netzwerk-)Konfigurations-Management kann es leicht passieren das Begriffe (bzw. unter-  #%
%# schiedliche Schreibweisen) wie:                                                                  #%
%# - Netzwerk-Konfigurationsmanagement                                                              #%
%# - Netzwerkkonfigurationsmanagement                                                               #%
%# - Netzwerk-Konfigurations-Management                                                             #%
%# oder auch nur                                                                                    #%        
%# - Konfigurations-Management                                                                      #%
%# - Konfigurationsmanagement                                                                       #%
%# Verwendet werden obwohl immer das selbe (der selbe Managementbereich) gemeint ist. Diese Inkon-  #%
%# sistenz führt zu Verwirrung des Lesers, dem nicht klar ist ob ein und das selbe oder etwas an-   #%
%# deres geweint ist. Darüber hinaus ist es kein guter Stiel.                                       #%
%#                                                                                                  #%
%# Aus diesem Grund wird der Begriff der eine Komponente, ein System etc. beschreibt einmal ein-    #%
%# deutig festgelegt und dann ausschließlich dieser verwendet.                                      #%
%#                                                                                                  #%
%# Verwendung:                                                                                      #%
%# Eine variable die ein Begriff darstellt wir wie folgt definiert:                                 #%
%# \newcommand{\terKM}{Konfigurations-Management}                                                   #%
%# Im Text verwendet wird die Variable (der Begriff) mit:                                           #%
%# \terKM{}                                                                                         #%
%# Dazu muss dieses Dokument einmal via %###########################################################################################################################################
%### TERMINILOGIE FESTLEGUNG ###############################################################################################################
%###########################################################################################################################################

%### Beschreibung ###################################################################################%
%# Dieses Dokument dient zur Festlegung der Terminoligie. Hier können Begriffe die bestimmte Kompo- #%
%# nenten, Systeme oder Sachverhalte beschreiben festgelegt und Variablen zugeordnet werden. Die    #%
%# Verwendung dieser Variablen in der Ausarbeitung soll sicherstellen das für eine Komponente, ein  #%
%# System oder Sachverhalt immer der selbe Begriff verwendet wird.                                  #%
%#                                                                                                  #%
%# Beispiel:                                                                                        #%
%# Für das (Netzwerk-)Konfigurations-Management kann es leicht passieren das Begriffe (bzw. unter-  #%
%# schiedliche Schreibweisen) wie:                                                                  #%
%# - Netzwerk-Konfigurationsmanagement                                                              #%
%# - Netzwerkkonfigurationsmanagement                                                               #%
%# - Netzwerk-Konfigurations-Management                                                             #%
%# oder auch nur                                                                                    #%        
%# - Konfigurations-Management                                                                      #%
%# - Konfigurationsmanagement                                                                       #%
%# Verwendet werden obwohl immer das selbe (der selbe Managementbereich) gemeint ist. Diese Inkon-  #%
%# sistenz führt zu Verwirrung des Lesers, dem nicht klar ist ob ein und das selbe oder etwas an-   #%
%# deres geweint ist. Darüber hinaus ist es kein guter Stiel.                                       #%
%#                                                                                                  #%
%# Aus diesem Grund wird der Begriff der eine Komponente, ein System etc. beschreibt einmal ein-    #%
%# deutig festgelegt und dann ausschließlich dieser verwendet.                                      #%
%#                                                                                                  #%
%# Verwendung:                                                                                      #%
%# Eine variable die ein Begriff darstellt wir wie folgt definiert:                                 #%
%# \newcommand{\terKM}{Konfigurations-Management}                                                   #%
%# Im Text verwendet wird die Variable (der Begriff) mit:                                           #%
%# \terKM{}                                                                                         #%
%# Dazu muss dieses Dokument einmal via %###########################################################################################################################################
%### TERMINILOGIE FESTLEGUNG ###############################################################################################################
%###########################################################################################################################################

%### Beschreibung ###################################################################################%
%# Dieses Dokument dient zur Festlegung der Terminoligie. Hier können Begriffe die bestimmte Kompo- #%
%# nenten, Systeme oder Sachverhalte beschreiben festgelegt und Variablen zugeordnet werden. Die    #%
%# Verwendung dieser Variablen in der Ausarbeitung soll sicherstellen das für eine Komponente, ein  #%
%# System oder Sachverhalt immer der selbe Begriff verwendet wird.                                  #%
%#                                                                                                  #%
%# Beispiel:                                                                                        #%
%# Für das (Netzwerk-)Konfigurations-Management kann es leicht passieren das Begriffe (bzw. unter-  #%
%# schiedliche Schreibweisen) wie:                                                                  #%
%# - Netzwerk-Konfigurationsmanagement                                                              #%
%# - Netzwerkkonfigurationsmanagement                                                               #%
%# - Netzwerk-Konfigurations-Management                                                             #%
%# oder auch nur                                                                                    #%        
%# - Konfigurations-Management                                                                      #%
%# - Konfigurationsmanagement                                                                       #%
%# Verwendet werden obwohl immer das selbe (der selbe Managementbereich) gemeint ist. Diese Inkon-  #%
%# sistenz führt zu Verwirrung des Lesers, dem nicht klar ist ob ein und das selbe oder etwas an-   #%
%# deres geweint ist. Darüber hinaus ist es kein guter Stiel.                                       #%
%#                                                                                                  #%
%# Aus diesem Grund wird der Begriff der eine Komponente, ein System etc. beschreibt einmal ein-    #%
%# deutig festgelegt und dann ausschließlich dieser verwendet.                                      #%
%#                                                                                                  #%
%# Verwendung:                                                                                      #%
%# Eine variable die ein Begriff darstellt wir wie folgt definiert:                                 #%
%# \newcommand{\terKM}{Konfigurations-Management}                                                   #%
%# Im Text verwendet wird die Variable (der Begriff) mit:                                           #%
%# \terKM{}                                                                                         #%
%# Dazu muss dieses Dokument einmal via %###########################################################################################################################################
%### TERMINILOGIE FESTLEGUNG ###############################################################################################################
%###########################################################################################################################################

%### Beschreibung ###################################################################################%
%# Dieses Dokument dient zur Festlegung der Terminoligie. Hier können Begriffe die bestimmte Kompo- #%
%# nenten, Systeme oder Sachverhalte beschreiben festgelegt und Variablen zugeordnet werden. Die    #%
%# Verwendung dieser Variablen in der Ausarbeitung soll sicherstellen das für eine Komponente, ein  #%
%# System oder Sachverhalt immer der selbe Begriff verwendet wird.                                  #%
%#                                                                                                  #%
%# Beispiel:                                                                                        #%
%# Für das (Netzwerk-)Konfigurations-Management kann es leicht passieren das Begriffe (bzw. unter-  #%
%# schiedliche Schreibweisen) wie:                                                                  #%
%# - Netzwerk-Konfigurationsmanagement                                                              #%
%# - Netzwerkkonfigurationsmanagement                                                               #%
%# - Netzwerk-Konfigurations-Management                                                             #%
%# oder auch nur                                                                                    #%        
%# - Konfigurations-Management                                                                      #%
%# - Konfigurationsmanagement                                                                       #%
%# Verwendet werden obwohl immer das selbe (der selbe Managementbereich) gemeint ist. Diese Inkon-  #%
%# sistenz führt zu Verwirrung des Lesers, dem nicht klar ist ob ein und das selbe oder etwas an-   #%
%# deres geweint ist. Darüber hinaus ist es kein guter Stiel.                                       #%
%#                                                                                                  #%
%# Aus diesem Grund wird der Begriff der eine Komponente, ein System etc. beschreibt einmal ein-    #%
%# deutig festgelegt und dann ausschließlich dieser verwendet.                                      #%
%#                                                                                                  #%
%# Verwendung:                                                                                      #%
%# Eine variable die ein Begriff darstellt wir wie folgt definiert:                                 #%
%# \newcommand{\terKM}{Konfigurations-Management}                                                   #%
%# Im Text verwendet wird die Variable (der Begriff) mit:                                           #%
%# \terKM{}                                                                                         #%
%# Dazu muss dieses Dokument einmal via \input{PATH\terminology} in das Textdokument eingebunden    #%
%# werden.                                                                                          #%
%# Manchmal empielt es sich die Trennstellen des Wortes mit anzugeben z.B.:                         #%
%# \newcommand{\KM}{Konfigu\-rations-Manage\-ment}                                                  #%
%####################################################################################################%

%### Variablen/Begriffs Deklaration ########################################################################################################

%Beispiele:
\newcommand{\LE}{LEGO}
\newcommand{\LM}{LEGO MINDSTORMS}


%###########################################################################################################################################
%### EOF ###################################################################################################################################
%########################################################################################################################################### in das Textdokument eingebunden    #%
%# werden.                                                                                          #%
%# Manchmal empielt es sich die Trennstellen des Wortes mit anzugeben z.B.:                         #%
%# \newcommand{\KM}{Konfigu\-rations-Manage\-ment}                                                  #%
%####################################################################################################%

%### Variablen/Begriffs Deklaration ########################################################################################################

%Beispiele:
\newcommand{\LE}{LEGO}
\newcommand{\LM}{LEGO MINDSTORMS}


%###########################################################################################################################################
%### EOF ###################################################################################################################################
%########################################################################################################################################### in das Textdokument eingebunden    #%
%# werden.                                                                                          #%
%# Manchmal empielt es sich die Trennstellen des Wortes mit anzugeben z.B.:                         #%
%# \newcommand{\KM}{Konfigu\-rations-Manage\-ment}                                                  #%
%####################################################################################################%

%### Variablen/Begriffs Deklaration ########################################################################################################

%Beispiele:
\newcommand{\LE}{LEGO}
\newcommand{\LM}{LEGO MINDSTORMS}


%###########################################################################################################################################
%### EOF ###################################################################################################################################
%########################################################################################################################################### in das Textdokument eingebunden    #%
%# werden.                                                                                          #%
%# Manchmal empielt es sich die Trennstellen des Wortes mit anzugeben z.B.:                         #%
%# \newcommand{\KM}{Konfigu\-rations-Manage\-ment}                                                  #%
%####################################################################################################%

%### Variablen/Begriffs Deklaration ########################################################################################################

%Beispiele:
\newcommand{\LE}{LEGO}
\newcommand{\LM}{LEGO MINDSTORMS}


%###########################################################################################################################################
%### EOF ###################################################################################################################################
%###########################################################################################################################################