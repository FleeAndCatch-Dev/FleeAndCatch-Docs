\setcounter{page}{1}
\ofoot{\pagemark}

\section{Einleitung}

Heutzutage werden viele Arbeitsschritte im Industrie-, Dienstleistungs- und Agrarsektor von Maschinen verrichtet, da diese eine effizientere Arbeit leisten und weniger Kosten als Menschen verursachen. Die zum Einsatz kommenden Maschinen arbeiten dabei meist mit Menschen, oder anderen Maschinen für ein gemeinsames Ziel zusammen. Um dabei einen optimalen Arbeitsablauf zu schaffen wird auf die Theorie des Schwarmverhaltens gesetzt, indem die zusammen agierenden Maschinen sich gegenseitig im Arbeitsablauf unterstützen. Diese Verhaltensstrukturen stammen meist aus der Tierwelt, wie am Beispiel von Fischschwärmen, Ameisen oder Bienen, wobei jedes Individuum des Schwarms seine Aufgaben zum gemeinsamen Überleben erfüllt.\\
In diesem Projekt werden die grundlegenden Verhaltensstrukturen von Tieren aufgegriffen, welche für ein Verfolgungsszenario, wie am Beispiel von autonom fahrenden Autos möglich ist.

\begin{comment}
Um den Ablauf der Arbeitsschritte, der zum Einsatz kommenden Maschinen zu optimieren, setzen viele auf ein Schwarmverhalten, welches eine Kommunikation

	In diesem Projekt werden diese Verhaltensmuster aus dem Tierreich aufgegriffen und anhand eines Verhaltensszenarios mit Kleinrobotern verwirklicht, die autonom agieren und kommunizieren, um zusammen ihr Ziel zu erreichen. Dabei sollen Konzepte, sowie Algorithmen für Schwarmroboter entstehen, die auch auf andere Szenarien angewendet werden können.\\
\end{comment}

\newpage
\subsection{Ausgangslage}

Die Aufgangslage stellen die einzelnen ROboter dar, die unter der Mindstorm SOftware funktionieren und dadurch einzeln agieren können. Dabei ist keinerlei Kommunikation zwischen den einzelnen ROboter möglich und somit keinerlei Realisierung für ein Schwarmverhalten. Die Roboter haben verschiedene Möglichkeiten durch WLAN, Bluetooth zur KOmmunikation, wobei keines von LEGO MIndstorm dirket zur Verfügung steht. Diese Schnittstellen werden lediglich zur Nutzerinteraktion, sowie dem drahtlosen starten und debuugen der Software genutzt.

\subsection{Zielsetzung}

Das Ziel dieser Studienarbeit stellt die Implementierung eines komplexen Schwarmverhalten mit einem Verfolgungsszenario zwischen Kleinrobotern. Dabei soll ein KOmmunikationssystem aufgebaut werden, auf dessen Grundlage die Komponenten funktionieren. Dabei sollen entsprechende Grundaktionen, wie Bewegen der ROboter und Sensorikansteuerung realisiert werden.

\subsection{Erwartetes Ergebnis}

Erwartet wird ein Ergebnis, indem der Nutzer des Systems über eine mobile App mithilfe einer zentralen STeuereinheit einen beliebigen Kontext für ein ausgewähltes Schwarmverhalten starten kann. Dabei soll eine abstrakte IMplementierung beachtet werden, indem verschiedene RObotertypen und Szenarien für Schwarmverhalten dargestellt sind, um wenn erwünscht ein komplexes Szanrio mit unterschiedlichen RObotern zu starten um definierte AUfgaben zu lösen. Dabei ist vor allem die Implementierung der Grundsteuerung mitsammt der Kommunikation des Systems entscheident.