\setcounter{page}{1}
\ofoot{\pagemark}

\section{Einleitung}

Heutzutage werden viele Arbeitsschritte im Agrar-, Industrie- und Dienstleistungssektor von Computern und Roboter verrichtet, da diese eine effizientere Arbeit leisten und weniger Kosten als Menschen verursachen. Diese Technologien erfuhren in den letzten Jahren einen immer stärkeren Wandel durch laufende technische Innovationen, die deren Arbeitsablauf verbessern und somit produktiver gestalten.
Eine dieser Technologien stellt das theoretische Konzept eines Schwarmverhaltens dar, welches Unternehmen zur Kooperation verschiedener Computer und Roboter einsetzen. Dies ermöglicht die gegenseitige Unterstützung der Komponenten und somit einen geteilten Arbeitsablauf, um die jeweiligen Stärken zu nutzen. Diese Verhaltensstrukturen stammen meist aus der Tierwelt, wie am Beispiel von Fischschwärmen, Ameisen oder Bienen, wobei jedes Individuum des Schwarms seine Aufgeben für das Überleben des Schwarms erfüllt.\\
In diesem Projekt werden die grundlegenden Verhaltensstrukturen von in Schwärmen lebenden Tieren zur Umsetzung eines Verfolgungsszenarios, wie am Beispiel eines autonom fahrenden Autos, aufgegriffen. Dabei werden feste Regeln anhand von Benutzerszenarios definiert, auf welche der Roboterschwarm entsprechend der Nutzerhandlungen reagiert. Dies wird durch ein zentrales Kommunikationssystem umgesetzt, an welches alle teilnehmenden Komponenten angeschlossen sind.\\
Die Einsatzgebiete dieses Projektes sind dabei entsprechend groß und kann überall eingesetzt werden, wo technische Komponenten für ein gemeinsames Produkt zusammenarbeiten müssen. Beispiele hierfür ist die Optimierung von Produktionsanlagen, oder Verkehrsführungen in Form von autonom fahrenden Autos.

\newpage
\subsection{Ausgangslage}

Die Ausgangslage des Projektes stellen verschiedene Computer und Roboter ohne intelligentes Kommunikationssystem dar, welche einen Mehrwert durch eine Dienstleistung oder die Bearbeitung eines Produktes erwirtschaften. In diesem Projekt wird dies durch den Einsatz von LEGO Mindstorm EV3 Roboter dargestellt, welche mittels Struktur orientierter Programmierung verwendet werden können. Diese verfügen über verschiedene Schnittstellen sowie Sensorik, um mit ihrer Umwelt interagieren zu können und stellen damit die Basis eines Schwarmes dar.

\begin{itemize}
	\item LEGO Mindstorm EV3
	\item LEGO Mindstorm Sensorik
	\item Kabellose Netzwerkschnittstelle
\end{itemize}

\subsection{Zielsetzung}

Das Ziel dieser Studienarbeit stellt die Implementierung eines Schwarmverhaltens mit dem Fokus auf ein Verfolgungsszenario zwischen Kleinrobotern dar. Dabei soll ein zentrales Kommunikationssystem aufgebaut werden, auf dessen Grundlage die Komponenten miteinander interagieren. Die Roboter enthalten hierbei Grundfunktionen, die über Kommandos des Kommunikationssystems angesprochen werden, um die definierten Benutzerszenarios auszuführen und den Roboter zu bewegen.

\begin{itemize}
	\item Zentrales Kommunikationssystem
	\item Grundlegende Steuerungsfunktionen
	\item Abbildung von Schwarmverhalten
	\item Darstellung der aktuellen Daten
\end{itemize}

\begin{comment}
	\subsection{Erwartetes Ergebnis}
	
	Erwartet wird ein Ergebnis, indem der Nutzer des Systems über eine mobile App mithilfe einer zentralen STeuereinheit einen beliebigen Kontext für ein ausgewähltes Schwarmverhalten starten kann. Dabei soll eine abstrakte IMplementierung beachtet werden, indem verschiedene RObotertypen und Szenarien für Schwarmverhalten dargestellt sind, um wenn erwünscht ein komplexes Szanrio mit unterschiedlichen RObotern zu starten um definierte AUfgaben zu lösen. Dabei ist vor allem die Implementierung der Grundsteuerung mitsammt der Kommunikation des Systems entscheident.
\end{comment}