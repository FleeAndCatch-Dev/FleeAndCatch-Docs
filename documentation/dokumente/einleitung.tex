\setcounter{page}{1}
\ofoot{\pagemark}

\section{Einleitung}

Heutzutage werden viele Arbeitsschritte im Industrie-, Dienstleistungs- und Agrarsektor von Maschinen verrichtet, da diese eine effizientere Arbeit leisten und weniger Kosten als Menschen verursachen. Diese Technik erfuhr in den letzten Jahren eine immer stärkere Verbesserung durch einen verbesserten Arbeitsablauf und technische Inovationen. Um diesem laufenden Wandel gerecht zu werden setzen immer mehr Unternehmen auf Industrie 4.0, was aktuell zu einem regelrechten Schlagwort im technischen Bereich geworden ist.


 Die zum Einsatz kommenden Maschinen arbeiten dabei meist mit einfachster Technik für die Umsetzung 
 
 
 Um dabei einen optimalen Arbeitsablauf zu schaffen wird auf die Theorie des Schwarmverhaltens zurückgegriffen, indem die zusammen agierenden Maschinen sich gegenseitig im Arbeitsablauf unterstützen und die gegenseitigen Schwächen ausgleichen. Diese Verhaltensstrukturen stammen meist aus der Tierwelt, wie am Beispiel von Fischschwärmen, Ameisen oder Bienen, wobei jedes Individuum des Schwarms seine Aufgaben für ein gemeinsames Überleben des Schwarmes erfüllt.\\

\begin{comment}

	\begin{itemize}
		\item Verhaltensstrukturen von Tieren
		\item Verwirklichung anhand kleinroboter
		\item Verfolgungsszenario
		\item standard methoden zum autonomen Fahren
		\item Reaktion durch Kommandos
		\item Einsatz: STauvermeidung, optimale verkehrsführung, abschleppen, ...
		\item 
	\end{itemize}

	In diesem Projekt werden die grundlegenden Verhaltensstrukturen von in Schwärmen lebenden Tieren aufgegriffen, welche für ein Verfolgungsszenario, wie am Beispiel eines autonom fahrenden Autos, benötigt werden. Diese werden anhand eines Verfolgungsszenario von verschiedenen Kleinrobotern verwirklicht, indem verschiedene gundlegende Methoden zur Aktion von robotern implementiert wird. Der roboter soll sich damit mittels auslösender Kommandos autonom agieren. DIeses Szenario kann für viele aktuelle Punkte eingesetzt werde, wie zur STauvermeidung und einer optimalen Verkehrsführung.
\end{comment}

\newpage
\subsection{Ausgangslage}

Die Ausgangslage des Projektes stellen verschiedene Maschinen und Roboter ohne intelligentes System dar, welche einen Mehrwert durch eine Dienstleistung oder die Bearbeitung eines Produktes erwirtschaften. In diesem Projekt wird dies dargestellt durch den Einsatz von LEGO Mindstorm EV3 Roboter, welche mittels Struktur orientierter Programmierung verwendet werden können.

\begin{itemize}
	\item LEGO Mindstorm EV3
	\item LEGO Mindstorm Sensorik
	\item Kabellose Netzwerkschnittstelle
\end{itemize}

\begin{comment}
	Die Aufgangslage stellen die einzelnen ROboter dar, die unter der Mindstorm SOftware funktionieren und dadurch einzeln agieren können. Dabei ist keinerlei Kommunikation zwischen den einzelnen ROboter möglich und somit keinerlei Realisierung für ein Schwarmverhalten. Die Roboter haben verschiedene Möglichkeiten durch WLAN, Bluetooth zur KOmmunikation, wobei keines von LEGO MIndstorm dirket zur Verfügung steht. Diese Schnittstellen werden lediglich zur Nutzerinteraktion, sowie dem drahtlosen starten und debuugen der Software genutzt.
\end{comment}

\subsection{Zielsetzung}

Das Ziel dieser Studienarbeit stellt die Implementierung eines komplexen Schwarmverhalten mit einem Verfolgungsszenario zwischen Kleinrobotern. Dabei soll ein KOmmunikationssystem aufgebaut werden, auf dessen Grundlage die Komponenten funktionieren. Dabei sollen entsprechende Grundaktionen, wie Bewegen der ROboter und Sensorikansteuerung realisiert werden.

\begin{itemize}
	\item Kommunikationssystem
	\item Grundlegende Steuerungsfunktionen
	\item Abbildung von Schwarmverhalten
	\item Nutzereinsicht in erfasste Daten
\end{itemize}

\begin{comment}
	\subsection{Erwartetes Ergebnis}
	
	Erwartet wird ein Ergebnis, indem der Nutzer des Systems über eine mobile App mithilfe einer zentralen STeuereinheit einen beliebigen Kontext für ein ausgewähltes Schwarmverhalten starten kann. Dabei soll eine abstrakte IMplementierung beachtet werden, indem verschiedene RObotertypen und Szenarien für Schwarmverhalten dargestellt sind, um wenn erwünscht ein komplexes Szanrio mit unterschiedlichen RObotern zu starten um definierte AUfgaben zu lösen. Dabei ist vor allem die Implementierung der Grundsteuerung mitsammt der Kommunikation des Systems entscheident.
\end{comment}