\section{Evaluation und Reflektion}
In diesem Kapitel werden zum eine die erreichten Ergebnisse anhand der vorab definierten Zielsetzung evaluiert, zum anderen wird der Projektablauf reflektiert und die bei der Umsetzung des Projekts aufgetretenen Probleme sowie die dafür entwickelten Lösungen vorgestellt.
\medskip
\newline
Durch die Konzeption und Implementierung universeller Datenpakete in einem JSON-Format und einer entsprechend auf WLAN-Technologie basierenden Systemarchitektur in der das Backend die Kommunikationszentrale darstellt, ist es gelungen ein flexibles und abstraktes Kommunikationssystem 
für den Datenaustausch zwischen einer flexiblen Anzahl von Devices zu entwickeln. \\
Durch die Verwendung von TCP-Verbindungen für den Datentransfer ist ein zuverlässiger Datenaustausch garantiert, jedoch hat sich gezeigt, dass sich diese Art der Verbindung negativ auf die Latenz auswirkt und somit die Steuerung und Nachtstellung der verschiedenen Schwarmverhalten erschwert.
\\
Bei der Umsetzung der Robotersteuerung durch das Roboterprogramm wurde von Anfang an auf eine generische Programmierung geachtet, um das Programm flexibel und leicht erweiterbar zu halten. Durch die Verwendung eines Interfaces, welches entsprechende grundlegende Roboterfunktionen definiert, war es möglich die Steuerung des Roboters eines konkret implementierten Robotertyp zu abstrahieren. \\
Sämtliche Interaktion mit dem Roboter beschränkt sich dabei auf die im Interface definierten grundlegenden Funktionen. Dadurch wird es ermöglicht das Programm auf einfache Art und Weise zu erweitern. Für die Integration neuer Robotertypen ist lediglich eine neuen Roboter-Klasse notwendig, welche von dem Interface
erbt und die darin definierten Methoden für den neuen Roboter implementiert.
\medskip
\newline
Hauptaufgabe des Projekts ist die Nachtstellung von verschiedenen Szenarien zu Schwarmverhalten, wobei der Fokus auf der Realisierung eines Verfolgungsszenarios liegt. Zur Nachstellung dieser Szenarien war es vorab notwendig, eine grundlegende Steuerung des Roboters durch den Nutzer zu implementieren, um somit die Kontrolle des Leittiers eines Schwarms durch den Benutzer zu ermöglichen. \\
Anschließen erfolgte die Umsetzung der Datenfassung und deren Verteilung über das Backend 
zu den anderen Komponenten, so dass diese Aufgrund von Positionsdaten der anderen Sachwarmtiere 
ihre eigenen Bewegungen berechnen und durchführen konnten.
\\
Aufbauend auf diesen Grundprinzipien wurden anschließend die einzelnen Sachwarmszenarien 
umgesetzt. Die Szenarien Synchron in der sich die am Schwarm beteiligten Roboter in einer festen Struktur (z.B. Dreiecksformation) synchron sowie bewegen als auch das Follow Szenario, bei der 
sich die Roboter wie eine Art Schlange bewegen, wurden durch das zeitgleiche versenden identischer Steuerkommandos erreicht. Es hat sich gezeigt, dass die erfolgreiche Nachstellung dieser 
Szenarien sehr stark von der Verzögerung der Datenübertragung abhängt. Treffen die Pakete zu unterschiedliche Zeiten bei den Robotern ein, erfolgt die Steuerung zeitlich versetzt, was 
unweigerlich dazu führt, dass die Formation ihre Ordnung verliert.
\\
Im Gegensatz zu den Szenarien Synchron und Follow wurde das Flee Szenario, welches das Verfolgungsszenario darstellt über eine autonome Robotersteuerung basierend auf übertragenen
Positionsdaten realisiert. Durch die fortlaufenden und Fehler kompensierende Steuerung ist die
vorliegende Implementierung dieses Szenarios weit weniger durch eine schlechte Datenverbindung 
und hoher Latenz beeinflussbar, wie am Beispiel von anderen.
\medskip
\newline
Neben denen standardmäßigen Szenarien beinhaltete die Konzeption zusätzlich das Catch-Szenario, bei dem ein oder mehrere Roboter autonom nach einem Zufallsschema durch den Raum navigieren und der User diese durch Steuerung der App fangen muss. Ein solches Szenario erfordert jedoch eine vollständig autonome Steuerung des fliehenden Roboters durch einen auffälligen Algorithmus inklusive Hinderniserkennung und Berechnung der Position in Bezug auf die Spielfeldgrenzen. \\
Da dieses Szenario einen komplexen Ablauf mit einem erheblichen Implementierungsaufwand darstellt, war es aufgrund der beschränkten Zeit leider nicht möglich dieses zur realisieren.
\bigskip
\newline
Während des Projektes traten verschiedene Schwierigkeiten und Komplikationen auf, welche dieses auf unterschiedliche Weise beeinflussten. Hier eine Auflistung der bedeutenden Problemfälle:

\begin{itemize}
	\item Keine entsprechende Unterstützung von Bluetooth
	\item Latenz der Datenübertragung
	\item Unübersichtliche Struktur der Implementierung
	\item Verhalten des EV3 bzw. des leJOS-Systems
	% vergleich ziele, ergebnis
\end{itemize}

\noindent
 Zu Beginn des Projektes war für die Verbindung und den Datenaustausch zwischen Robotern und dem Backend eine Bluetooth-Verbindung konzipiert. Diese konnte wegen verschiedenen Problemen nicht umgesetzt werden. Ein Hauptproblem stellte dabei die komplexen Implementierung von Bluetooth auf den Komponenten sowie einer mangelhaften Dokumentation der Bibliotheken dar. \\
 Um dem Problem zu begegnen wurde daher ein Wechsel auf eine Netzwerkkommunikation (WLAN) mit TCP veranlasst, wie sie bereits zwischen App und Backend besteht. Dabei konnte die Implementierung teilweise übernommen werden, so dass diese nur an wenigen Stellen angepasst werden musste.\\

\noindent
Durch eine Verzögerung des Datenaustauschs, die mit dem Versenden der Kommandos in der Implementierung des Roboters auftrat, führte zu einer Verzögerung der Steuerung, welche die Nachstellung der geforderten Szenarien erschwerte. Dies wurde durch das Einfügen eines Initialisierungskommandos zum Teil kompensiert. Dabei werden laufend Kommandos zwischen den beiden Parteien hin- und hergesendet, wodurch dem Thread für die Datenübertragung mehr CPU-Zeit eingeräumt wird und die damit verbundene Verzögerung so gut wie möglich reduziert wird.\\

\noindent
Da eine Implementierung zwangsläufig zu Beginn eines Projektes etwas unstrukturiert ablief, kam es dabei zu einer Unübersichtlichkeit des Quellcodes. Um dies zu Verbessern wurde das \gls{mvvm} Design Pattern in der \gls{app} integriert, um in den einzelnen Teilen unkomplizierte Abänderungen des Quellcodes vorzunehmen und diesen so zu strukturieren.\\

\noindent
Da die EV3 Roboter auf sehr einfacher Hardware basieren und eigentlich als Kinderspielzeug gedacht sind, bringen diese einige technische Probleme mit sich, wenn man diese effizient einsetzen will. Beispielsweise bereitet die Installation eines Betriebssystems einige Schwierigkeiten, da dieses mittels einer SD-Karte auf den Roboter kopiert wird, um Anwendungen mittels Hochsprachen auszuführen zu können. Dies schlägt oft fehl und benötigt dadurch viel Zeit und Geduld. \\
Anderseits zeigen sich Probleme bei der Ausführung von Programmen zur Steuerung der Roboter, wobei vermutlich ein Überlauf der Daten bei bestimmten Methoden auftritt der dazu führt, dass sich der Roboter unkontrolliert verhält.\\