\section{Evaluation}

Durch die Durchführung dieses Projektes zum Thema \glqq{}Konzeption und Implementierung eines Schwarmverhaltens von mobilen Kleinrobotern anhand eines Verfolgungsszenarios\grqq{} entstand eine Software, welche die gesetzten Ziele erfüllt.

Während des Projektes traten verschiedene Komplikationen auf, welche diese auf unterschiedliche Weise beeinflussten. Hier eine Auflistung der bedeutenden Problemfälle:

\begin{itemize}
	\item Keine Funktionalität von Bluetooth
	\item Verzögerung der Steuerung
	\item Unübersichtliche Struktur der Implementierung
	\item Verhalten des EV3
	% vergleich ziele, ergebnis
\end{itemize}

\noindent
Da zu Beginn des Projektes zwischen den Robotern und dem Backend eine Bluetooth-Verbindung angedacht war, konnte diese leider nicht umgesetzt werden. Das Problem dabei lag an der komplexen Implementierung von Bluetooth auf den Komponenten sowie einer mangelhaften Dokumentation der Bibliotheken. Daher wurde ein Wechsel auf eine Netzwerkkommunikation mit TCP entschieden, wie sie bereits zwischen App und Backend besteht. Dabei konnte die Implementierung teilweise übernommen werden und musste an wenigen stellen angepasst werden.\\

\noindent
Durch einer Verzögerung der Steuerung, die mit dem Versenden der Kommandos im Roboter auftrat, erzeugte eine Verzögerung der Steuerung, welche dieses erschwert. Dies wurde durch das Einfügen eines Initialisierungskommandos bei der Anmeldung des Systems gelöst. Dabei werden laufend Kommandos zwischen den beiden Parteien hin- und hergesendet, damit die \gls{cpu}-Laufzeit im System des ROboter entsprechend erhöht wird und die damit verbundene Verzögerung so gut wie möglich verschwindet.\\

\noindent
Da eine Implementierung zwangsläufig zu Beginn eines Projektes etwas chaotisch abläuft, kommt es dabei zu einer Unübersichtlichkeit des Quellcode. Um dies zu Verbessern wurde das \gls{mvvm} Design Pattern in der \gls{app} integriert, um in den einzelnen Teilen unkomplizierte Abänderungen des Quellcodes vorzunehmen.\\

\noindent
Da die EV3 Roboter sehr einfach gestrickte Kleinroboter darstellen und eigentlich als Kinderspielzeug gedacht sind, bringen diese einige technische Probleme mit sich, wenn man diese effizient einsetzen will. Beispielsweise bereitet die Installation des Betriebsystems eineige Schwireigkeiten, da dieses mittels einer SD-Karte auf den Roboter kopiert wird, um Anwendungen mittels Hochsprachen auszuführen. Dies schlägt oft fehl und benötigt somit viel Zeit und Geduld. Anderseits zeigt sich dieses Problem in der Steuerung der Roboter, wobei ein Überlauf der Daten bei bestimmten Methoden entsteht und der ROboter sich mit voller Geschwindigkeit in bestimmte Richtungen bewegt, was er aber nicht machen sollte.\\

\noindent
Das Ergebnis dieses Projektes stellt ein sehr respektables Ergebnis dar, wobei auf eine sehr abstrakte Implementierung mittels Kommandos geachtet wurde. Dadurch ist dieses System in vielen Punkten erweiterbar und bietet die möglichkeiten für diverse neue Implementierungen von Logik. Dabei ist das Ziel vollstens erfüllt, wobei ein Verfolgungsszenario mittels eines Schwarmes von Kleinrobotern durchgeführt werden kann. Und wurde durch das Hinzufügen einer \gls{gui} der Desktopanwendung übertroffen, welche die Darstellung der erfassten Daten sämtlicher Komponenten ermöglicht.