%abbreviations
\newacronym[longplural={Applications}]{app}{App}{Application}
\newacronym{tcp}{TCP}{Transmission Control Protocol}
\newacronym{json}{JSON}{JavaScript Object Notation}
\newacronym{mvvm}{MVVM}{Model View ViewModel}
\newacronym{uml}{UML}{Unified Modeling Language}
\newacronym{cpu}{CPU}{Central Processing Unit}

%glossaries
\newglossaryentry{javascript}
{
	name=JavaScript,
	description={Skriptsprache zur Entwicklung dynamischer Internetseiten}
}
\newglossaryentry{typescript}
{
	name=TypeScript,
	description={Objektorientierte Programmiersprache für Webentwicklung}
}
\newglossaryentry{framework}
{
	name=Framework,
	description={Rahmen zur Programmierung mit verschiedenen Softwarekomponenten},
	plural=Frameworks
}
\newglossaryentry{java}
{
	name=Java,
	description={Objektorientierte Programmiersprache für plattformübergreifende Anwendungen}
}
\newglossaryentry{objectivc}
{
	name=Objective-C,
	description={Auf C basierte objektorientierte Programmiersprache}
}
\newglossaryentry{template}
{
	name=Template,
	description={Vorlage zur Implementierung}
}
\newglossaryentry{csharp}
{
	name=C\#,
	description={Objektorientierte Programmiersprache mit dem Schwerpunkt auf Typsicherheit}
}
\newglossaryentry{lejos}
{
	name=leJos,
	description={Java System zur Programmierung von LEGO Mindstorm}
}
\newglossaryentry{ev3}
{
	name=EV3,
	description={Kleinroboter der LEGO Mindstorm Serie}
}
\newglossaryentry{eclipse}
{
	name=Eclipse,
	description={Programmierumgebung für diverse Programmiersprachen und Frameworks}
}
\newglossaryentry{javafx}
{
	name=JavaFX,
	description={Framework zur Erstellung von Java Anwendungen}
}
\newglossaryentry{sqlite}
{
	name=SQLite,
	description={Bibliothek zur Erstellung einer lokalen relationalen Datenbank}
}

%dual entry
\newglossaryentry{gls-api} {
	name={Application Programming Interface},
	description={Programmierschnittstelle, die die Anbindung von Software ermöglicht},
}
\newacronym[see={[Glossary:]{gls-api}}]{api}{API}{Application Programming Interface\glsadd{gls-api}}
\newglossaryentry{gls-html}
{
	name=Hypertext Markup Language,
	description={Basiert auf XML und bestimmt den Aufbau einer Internetseite},
}
\newacronym[see={[Glossary:]{gls-html}}]{html}{HTML}{Hypertext Markup Language\glsadd{html}}
\newglossaryentry{gls-css}
{
	name=Cascading Style Sheets,
	description={Bestimmt das Design einer Internetseite}
}
\newacronym[see={[Glossary:]{gls-css}}]{css}{CSS}{Cascading Style Sheets\glsadd{css}}
\newglossaryentry{gls-xml}
{
	name=Extensible Markup Language,
	description={Ein auf Text basiertes Format zum Austausch von Informationen und Daten}
}
\newacronym[see={[Glossary:]{gls-xml}}]{xml}{XML}{Extensible Markup Language\glsadd{xml}}
\newglossaryentry{gls-ui}
{
	name=User Interface,
	description={Eine Nutzerschnittstelle, zur Interaktionen mit der Anwendung}
}
\newacronym[see={[Glossary:]{gls-ui}}]{ui}{UI}{User Interface\glsadd{ui}}
\newglossaryentry{gls-gui}
{
	name=Graphical User Interface,
	description={Eine grafische Nutzerschnittstelle, zur Interaktionen mit der Anwendung}
}
\newacronym[see={[Glossary:]{gls-gui}}]{gui}{GUI}{Graphical User Interface\glsadd{gui}}
\newglossaryentry{gls-aot}
{
	name=Ahead of Time,
	description={Übersetzt Programmcode vor der Ausführung in Maschinencode}
}
\newacronym[see={[Glossary:]{gls-aot}}]{aot}{AOT}{Ahead of Time\glsadd{aot}}
\newglossaryentry{gls-arm}
{
	name=Acorn RISC Machines,
	description={Mikroprozessordesign}
}
\newacronym[see={[Glossary:]{gls-arm}}]{arm}{ARM}{Acorn RISC Machines\glsadd{arm}}
\newglossaryentry{gls-il}
{
	name=Intermediate Language,
	description={Intermediate Language ist eine objektorientierte Assemblersprache}
}
\newacronym[see={[Glossary:]{gls-il}}]{il}{IL}{Intermediate Language\glsadd{il}}
\newglossaryentry{gls-jit}
{
	name=Just-in-Time,
	description={Der entsprechende Quellcode wird zur Laufzeit übersetzt}
}
\newacronym[see={[Glossary:]{gls-jit}}]{jit}{JIT}{Just-in-Time\glsadd{jit}}
\newglossaryentry{gls-dll}
{
	name=Dynamic Linked Library,
	description={Dynamische Bibliothek, die zur Laufzeit dem Programmcode hinzugefügt wird},
	plural=DLLs
}
\newacronym[see={[Glossary:]{gls-dll}}]{dll}{DLL}{Dynamic Linked Library\glsadd{dll}}
\newglossaryentry{gls-cli}
{
	name=Common Language Infrastructure,
	description={Spezifikation zur sprach- und plattformneutralen Entwicklung von Anwendungen}
}
\newacronym[see={[Glossary:]{gls-cli}}]{cli}{CLI}{Common Language Infrastructure\glsadd{cli}}
\newglossaryentry{gls-clr}
{
	name=Common Language Runtime,
	description={Laufzeitumgebung zur Ausführung von .NET Anwendungen}
}
\newacronym[see={[Glossary:]{gls-clr}}]{clr}{CLR}{Common Language Runtime\glsadd{clr}}