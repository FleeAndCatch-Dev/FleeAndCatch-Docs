%abbreviations
\newacronym[longplural={Applications}]{app}{App}{Application}

%glossaries
\newglossaryentry{javascript}
{
	name=JavaScript,
	description={JavaScript ist eine Skriptsprache zur Entwicklung dynamischer Internetseiten}
}
\newglossaryentry{typescript}
{
	name=TypeScript,
	description={TypeScript ist eine objektorientierte Programmiersprache von Microsoft basierend auf den ECMA-Script-6 Standards}
}
\newglossaryentry{framework}
{
	name=Framework,
	description={Ein Framework ist ein Rahmen zur Programmierung mit verschiedenen Softwarekomponenten},
	plural=Frameworks
}
\newglossaryentry{java}
{
	name=Java,
	description={}
}
\newglossaryentry{objectivc}
{
	name=Objective-C,
	description={}
}
\newglossaryentry{template}
{
	name=Template,
	description={}
}

%dual entry
\newglossaryentry{gls-api} {
	name={Application Programming Interface},
	description={Eine API ist eine Programmierschnittstelle, die die Anbindung von Software ermöglicht},
}
\newacronym[see={[Glossary:]{gls-api}}]{api}{API}{Application Programming Interface\glsadd{gls-api}}
\newglossaryentry{gls-html}
{
	name=Hypertext Markup Language,
	description={HTML basiert auf XML und bestimmt den Aufbau einer Internetseite},
}
\newacronym[see={[Glossary:]{gls-html}}]{html}{HTML}{Hypertext Markup Language\glsadd{html}}
\newglossaryentry{gls-css}
{
	name=Cascading Style Sheets,
	description={CSS bestimmt das Design einer Internetseite}
}
\newacronym[see={[Glossary:]{gls-css}}]{css}{CSS}{Cascading Style Sheets\glsadd{css}}
\newglossaryentry{gls-xml}
{
	name=Extensible Markup Language,
	description={XML ist ein auf Text basiertes Format zum Austausch von Informationen und Daten}
}
\newacronym[see={[Glossary:]{gls-xml}}]{xml}{XML}{Extensible Markup Language\glsadd{xml}}
\newglossaryentry{gls-ui}
{
	name=User Interface,
	description={Die UI ist eine Nutzerschnittstelle, für Interaktionen mit der Anwendung}
}
\newacronym[see={[Glossary:]{gls-ui}}]{ui}{UI}{User Interface\glsadd{ui}}
\newglossaryentry{gls-aot}
{
	name=Ahead of Time,
	description={Übersetzt Programmcode vor der Ausführung in Maschinencode}
}
\newacronym[see={[Glossary:]{gls-aot}}]{aot}{AOT}{Ahead of Time\glsadd{aot}}
\newglossaryentry{gls-arm}
{
	name=Acorn RISC Machines,
	description={Mikroprozessordesign der Firma Acorn}
}
\newacronym[see={[Glossary:]{gls-arm}}]{arm}{ARM}{Acorn RISC Machines\glsadd{arm}}
\newglossaryentry{gls-il}
{
	name=Intermediate Language,
	description={Intermediate Language ist eine objektorientierte Assemblersprache}
}
\newacronym[see={[Glossary:]{gls-il}}]{il}{IL}{Intermediate Language\glsadd{il}}
\newglossaryentry{gls-jit}
{
	name=Just-in-Time,
	description={Just-in-Time bedeutet, das der entsprechende Quellcode zur Laufzeit übersetzt wird}
}
\newacronym[see={[Glossary:]{gls-jit}}]{jit}{JIT}{Just-in-Time\glsadd{jit}}
\newglossaryentry{gls-dll}
{
	name=Dynamic Linked Library,
	description={Bezeichnet eine dynamische Bibliothek, die zur Laufzeit dem Programmcode hinzugefügt wird},
	plural=DLLs
}
\newacronym[see={[Glossary:]{gls-dll}}]{dll}{DLL}{Dynamic Linked Library\glsadd{dll}}