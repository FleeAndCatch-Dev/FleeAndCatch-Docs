\section{Projektdefinition}

Dieser Abschnitt beschäftigt sich mit der allgemeinen Definition der Studienarbeit.

\subsection{Gründung des Projektes}

Diese Studienarbeit wird sich mit dem Thema, \textbf{Experimente zum Schwarmverhalten(Kooperation) von mobilen Kleinrobotern}, beschäftigen. Bearbeiter dieser Studienarbeit sind hierbei Manuel Bothner und Simon Lang und wird von Prof. Hans-Jörg Haubner betreut.

\subsection{Festlegung des Projektziels}

Dieser Abschnitt beschäftigt sich mit dem Ziel der Studienarbeit und deren groben Verlauf.

\subsubsection{Titel}

Konzeption und Implementierung eines Sachwarmverhaltens von mobilen Kleinrobotern anhand eines Verfolgungsszenarios.\\

\subsubsection{Kurzbeschreibung}

Durch das Projekt soll gezeigt werden, in wie weit durch die Programmierung mehrerer interagierender, mobiler Kleinroboter verhaltenstypischer Situationen von natürlichen Schwärmen nachgestellt werden können. Dieses Studienarbeit zeigt dies anhand mehrerer Kleinroboter von LEGO Mindstorm, einer App zur Steuerung des Szenarios, sowie einer zu implementierenden Schnittstelle der beiden Komponenten.\\
Auf der App soll der Nutzer die Möglichkeit besitzen unter verschiedenen Varianten des Verfolgungsszenarios auszuwählen, dies beinhaltet als Beispiel eine direkte Steuerung eines Roboters über den Touchscreen, oder einer intelligenten autonomen Verfolgung. Die Schnittstelle zwischen beiden Komponenten dient der allgemeinen Steuerung des Szenarios, dass eine Erweiterung um mehrerer Clients, als auch Roboter ermöglichen soll, was das Verfolgungsszenario wesentlich interessanter für verschiedene Algorithmen macht. Die Roboter erhalten über eine drahtlose Schnittstelle entsprechende Daten zur Navigation, um sich in seiner Umgebung zu bewegen und am Verfolgungsszenario teilzunehmen.\\